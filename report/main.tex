\documentclass[a4paper]{report}

\usepackage[a4paper, top=2cm, bottom=2cm, left=2cm, right=2cm]{geometry}
\usepackage[T1, T2A]{fontenc}
\usepackage[fontsize=12pt]{fontsize}
\usepackage{anyfontsize}
\usepackage[english, russian]{babel}
\usepackage[utf8]{inputenc}
\usepackage{hyperref}
\usepackage{fancyhdr}
\usepackage{sectsty}
\usepackage{amsthm}
\usepackage{amsmath}
\usepackage{amsfonts}
\usepackage{physics}
\usepackage{cleveref}
\usepackage{titlesec}
\usepackage{indentfirst}
\usepackage{pgfplots}
\pgfplotsset{compat=newest}
\usepackage{enumitem}
\usepackage{ulem}
\usepackage{float}
\usepackage{cmap}
\usepackage{pscyr} % For Times New Roman
\usepackage{lastpage}
\usepackage{tabularx}

\renewcommand{\rmdefault}{ftm}

\titleformat{\chapter}[display]
{\normalfont\huge\bfseries}
{\chaptertitlename\ \thechapter\bigskip}{0pt}{\Huge}

\titleformat{\section}
{\normalfont\huge\bfseries}
{\thesection}{1em}{\huge}

\titleformat{\subsection}
{\normalfont\LARGE\bfseries}
{\thesubsection}{1em}{\LARGE}

\hypersetup{
	colorlinks=true,
	urlcolor=cyan,
	linkcolor=cyan,
}

\newcommand{\ProjectName}{<<Менеджер расписания>>}
\newcommand{\ProjectVersion}{1.0}
\newcommand{\Comment}[1]{{\color{blue}#1}}

\fancyhf{}
\fancyfoot[R]{\thepage\;(\pageref{LastPage})}

\begin{document}
	\pagestyle{empty}
	\begin{center}
		{\large Новосибирский Государственный Университет
		
		Факультет Информационных Технологий}
	\end{center}

	\begin{flushright}
		\bfseries{\huge Техническое описание проекта
		
		по курсу ООАД

		\bigskip

		\ProjectName
		
		\bigskip
		
		Студенты ФИТ НГУ
		
		Неретин Степан Иванович
		
		Кондренко Кирилл Павлович
		
		Группа 21203}
		
		\bigskip
		
		{\large Версия \ProjectVersion}
	\end{flushright}

	{\hypersetup{hidelinks}\tableofcontents\pagestyle{fancy}}

	\pagestyle{fancy}

	\chapter{Введение}
		\section{Цель}
			Данный документ представляет собой техническое описание проекта \ProjectName и содержит основные требования к разрабатываемой в рамках проекта программной системе и описание архитектуры программного решения.
		\section{Область действия}
			Документ разработан в рамках проекта \ProjectName на основе стандартного шаблона и предназначен для использования студентами ФИТ и преподавателями дисциплины ООАД
		\section{Определения и сокращения}
	
		\section{Ссылки}
		
		\section{Краткое описание}	
	\chapter{Предметная область проекта}
		Надстройка в виде телеграмм-бота или расширения для браузера, улучшение текущего представление расписания для студентов НГУ посредством гибкого редактирования (скрытия <<лишних>> предметов, добавления новых).
	
		\section{Существующие проблемы}
			У многих студентов НГУ в расписании стоят предметы, которые они не посещают (например, это <<предметы по выбору>>, из которых студент выбрал лишь один) или же нет предметов, на которые они ходят (например, студент может по своему желанию посещать семинары других преподавателей дополнительно к своим).
			
			Решением этого было бы составление университетом индивидуального расписания для каждого из студентов, однако данное решение не реализуется университетом. Поэтому единственным решением в данном случае остаётся создание такой системы, с помощью которой студент сам бы мог редактировать своё расписание, при этом все изменения бы оставались и отображались там, где студент чаще всего смотрит своё расписание. 
		\section{Предполагаемое решение}
			Ввиду того что чаще всего студенты смотрят своё расписание через сайт в браузере и в приложении от университета для телефона, решением будет создать расширение, работающее в любом современном браузере, и телеграмм-бота, которые должны быть взаимозаменяемыми и работать друг с другом.
	\chapter{Требования к программному решению}
		\section{Роли}
		
		\section{Функциональные требования для роли Роль1}
			\subsection{<Use Case Name 1>}
		
			\subsection{<Use Case Name 2>}
		\section{Функциональные требования для роли Роль2}
			\subsection{<Use Case Name 1>}
		
			\subsection{<Use Case Name 2>}
			
		\section{Нефункциональные требования}
	\chapter{Обзор архитектуры}
		\section{Компонентная модель системы}
			\subsection{Компонент 1}
				выфщвоыфлвоыфдловыфдловыфдлвфоыдл
			\subsection{Компонент 2}
		\section{Компоненты сторонних производителей}
		
		\section{Схема развёртывания приложения}
	\chapter{Допущения и ограничения}
	
	\chapter{Известные проблемы}
	
	\chapter*{Лист регистрации изменений}
		\begin{tabularx}{\textwidth}{ 
				| >{\centering\arraybackslash}X 
				| >{\centering\arraybackslash}X 
				| >{\centering\arraybackslash}X
				| >{\centering\arraybackslash}X |}
			\hline
			\textbf{Дата} & \textbf{Версия} & \textbf{Описание} & \textbf{Автор} \\
			\hline
			&&& \\
			\hline
		\end{tabularx}
	\chapter*{Лист регистрации проверок}
		\begin{tabularx}{\textwidth}{ 
				| >{\centering\arraybackslash}X 
				| >{\centering\arraybackslash}X 
				| >{\centering\arraybackslash}X
				| >{\centering\arraybackslash}X |}
			\hline
			\textbf{Дата} & \textbf{Версия} & \textbf{Описание} & \textbf{Автор} \\
			\hline
			&&& \\
			\hline
		\end{tabularx}
\end{document}
