\chapter{Предметная область проекта}
	Надстройка в виде <<телеграмм-бота>> или <<расширения для браузера>>, улучшение текущего представление расписания для студентов НГУ посредством гибкого редактирования (скрытия <<лишних>> предметов, добавления новых).

	\section{Существующие проблемы}
		У многих студентов НГУ в расписании стоят предметы, которые они не посещают (например, это <<предметы по выбору>>, из которых студент выбрал лишь один) или же нет предметов, на которые они ходят (например, студент может по своему желанию посещать семинары других преподавателей дополнительно к своим).

		Решением этого было бы составление университетом индивидуального расписания для каждого из студентов, однако данное решение не реализуется университетом. Поэтому единственным решением в данном случае остаётся создание такой системы, с помощью которой студент сам бы мог редактировать своё расписание, при этом все изменения бы оставались и отображались там, где студент чаще всего смотрит своё расписание. 
	\section{Предполагаемое решение}
	Ввиду того что чаще всего студенты смотрят своё расписание через сайт в браузере и в приложении от университета для телефона, решением будет создать <<расширение для браузера>>, работающее в любом современном браузере, и <<телеграмм-бота>>, которые должны быть взаимозаменяемыми и работать друг с другом.
