\chapter{Известные проблемы}
	\label{chapter6}
	\phantomsection

	\section{Отсутствие поддержки JWT}
		\begin{tabularx}{\textwidth}{
				| >{\centering\arraybackslash\hsize=4cm}X
				| >{\centering\arraybackslash}X
				|}
			\hline
			\textbf{Проблема} &  Для доступа к данным API бэкенда не требуется никакая авторизация \\\hline
			\textbf{Ранг} &  3 (низкий) \\\hline
			\textbf{Влияние на проект} &  Любой пользователь интернета может получать данные с бэкенда, что увеличивает нагрузку на него. Как следствие, оба фронтенда могут начать работать медленнее \\\hline
			\textbf{Пути решения} &  Добавить поддержку JWT, чтобы доступ к бэкенду имелся лишь у авторизованных приложений (только у расширения, и только у бота), а не у любого пользователь интернет, знающего адрес бэкенда в сети интернет \\\hline
		\end{tabularx}
	\section{Отсутствие защиты от DOS и DDOS-атак}
		\begin{tabularx}{\textwidth}{
				| >{\centering\arraybackslash\hsize=4cm}X
				| >{\centering\arraybackslash}X
				|}
			\hline
			\textbf{Проблема} &  У бэкенда нет никакой защиты от DOS или DDOS-атак, поэтому при любой такой атаке он перестанет обрабатывать запросы фронтендов \\\hline
			\textbf{Ранг} &  5 (средний) \\\hline
			\textbf{Влияние на проект} &  При любой DOS или DDOS-атаке, оба фронтенда перестанут полноценно работать \\\hline
			\textbf{Пути решения} &  Использовать интернет сервисы для защиты от DOS и DDOS-атак. Например, CloudFlare \\\hline
		\end{tabularx}
	\section{Сохранение состояние пользователей телеграмм не в базе данных}
		\begin{tabularx}{\textwidth}{
				| >{\centering\arraybackslash\hsize=4cm}X
				| >{\centering\arraybackslash}X
				|}
			\hline
			\textbf{Проблема} &  Текущее состояние пользователей (не их расписание) телеграмм не сохраняется в базу данных, а хранится локально на том устройстве, на котором запущен телеграмм-бот \\\hline
			\textbf{Ранг} &  5 (средний) \\\hline
			\textbf{Влияние на проект} &  При большом количестве пользователей телеграмм-бота, может закончится память на устройстве, на котором запущен телеграмм-бот, а значит он перестанет обрабатывать запросы новых пользователей или вовсе прекратит работу \\\hline
			\textbf{Пути решения} &  Сохранять состояние пользователей в какую-нибудь удалённую базу данных \\\hline
		\end{tabularx}
	\section{Невозможность работы фронтендов без бэкенда}
		\begin{tabularx}{\textwidth}{
				| >{\centering\arraybackslash\hsize=4cm}X
				| >{\centering\arraybackslash}X
				|}
			\hline
			\textbf{Проблема} &  Если не работает бэкенд, то оба фронтенда не могут полноценно работать \\\hline
			\textbf{Ранг} &  7 (Высокий) \\\hline
			\textbf{Влияние на проект} &  При критических проблемах на стороне фронтенда оба фронтенда не могут полноценно работать \\\hline
			\textbf{Пути решения} &  Избавиться от бэкенда и осуществлять парсинг сайта с расписанием НГУ на каждом фронтенде. Однако это приведёт к дублированию логики и потенциальному расхождении спецификации. Также при добавлении нового фронтенда, эту логику придётся заново реализовывать на нём (если он будет написан на языке программирования, отличном от уже существующих на фронтендах)  \\\hline
		\end{tabularx}
	\section{Зависимость бэкенда и расширения браузера от вёрстки сайта с расписанием НГУ}
		\begin{tabularx}{\textwidth}{
				| >{\centering\arraybackslash\hsize=4cm}X
				| >{\centering\arraybackslash}X
				|}
			\hline
			\textbf{Проблема} &  Бэкенд и расширение браузера напрямую зависят от вёрстки сайта с расписанием НГУ \\\hline
			\textbf{Ранг} &  8 (Высокий) \\\hline
			\textbf{Влияние на проект} & Если у сайта с расписанием НГУ поменяется вёрстка, то для полноценной работы системы придётся изменять бэкенд и отображение страниц в расширении \\\hline
			\textbf{Пути решения} &  Использовать API, которое используют сайты с расписанием НГУ, однако данное API если и существует, то не является открытым \\\hline
		\end{tabularx}
