\chapter{Введение}
	\label{1}
	\phantomsection

	\section{Цель}
		Данный документ представляет собой техническое описание проекта \ProjectName и содержит основные требования к разрабатываемой в рамках проекта программной системе и описание архитектуры программного решения.
	\section{Область действия}
		Документ разработан в рамках проекта \ProjectName на основе стандартного шаблона и предназначен для использования студентами ФИТ и преподавателями дисциплины ООАД.
	\section{Определения и сокращения}
		\begin{tabularx}{\textwidth}{
				| >{\centering\arraybackslash\hsize=5cm}X
				| >{\centering\arraybackslash}X
				|}\hline
			\textbf{Термин} & \textbf{Описание} \\\hline
			Предметы по выборы & Предметы, разделенные по блокам, причем в каждом блоке студент расставляет их по приоритетам, но в конечном итоге из каждого блока для студента определён \textbf{лишь один} предмет \\\hline
			Телеграмм-бот & Приложение внутри приложения <<Telegram>>, которое позволяет совершать определённые действия в чате с ним \\\hline
			Расширение для браузера & Компьютерная программа, расширяющая функциональные возможности браузера \\\hline
		\end{tabularx}
	\section{Ссылки}

	\section{Краткое описание}
		Содержание данного документа построено таким образом, чтобы дать ответ на следующие вопросы:
		\begin{itemize}
			\item Какие проблемы предметной области должен решать будущий программный продукт
			
			\item Посредством какой функциональности системы будут достигнуто решение проблем предметной области
			
			\item Какова архитектура программного решения
		\end{itemize}
	
		Описание предметной области и проблем, для решения которых предназначен будущий программный продукт, приведены в разделе \hyperref[chapter2]{2}.
	
		\bigskip
		
		Раздел \hyperref[chapter3]{3} содержит описание требований к программному решению, раздел \hyperref[chapter4]{4} --- описание архитектуры выбранного решения.
