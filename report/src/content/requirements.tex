\newcommand{\actor}[1]{\textit{<<#1>>}}

\chapter{Требования к программному решению}
	\label{chapter3}
	\phantomsection

	Данный раздел описывает требования к программной системе, разрабатываемой в рамках проекта \ProjectName.
		
	\section{Роли}
		Роль --- это что-то (например: другая система) или кто-то (например: человек) вне системы, которые взаимодействуют с ней. В предлагаемой к разработке системе идентифицированы следующие роли:
		\begin{enumerate}
			\item \actor{Пользователь браузера} --- человек, который зашёл на сайт с расписанием группы НГУ;
			
			\item \actor{Пользователь браузера с расписанием} -- пользователь браузера, для которого определено текущее расписание.
		\end{enumerate}
	
		\begin{figure}[h!]
			\centering
			\def\svgwidth{\columnwidth}
			\input{browser_UC.pdf_tex}
		\end{figure}
	\section{Функциональные требования для роли \actor{Пользователь браузера}}
		\subsection{Установить расписание}
			Как только \actor{Пользователь браузера} зашёл на сайт с расписанием группы, текущее расписание устанавливается как расписание этой группы. Однако если по какой-то причине это не удалось сделать, то пользоваться расширением не получится. Если удалось установить расписание, то данное расписание сохраняется в кэш браузера.
			
			Если \actor{Пользователь браузера} ранее уже заходил на сайт с расписанием данной группы, то текущее расписание восстановится из кэша браузера.
	\section{Функциональные требования для роли \actor{Пользователь браузера с расписанием}}
		\subsection{Осуществить операцию над расписанием}
			\actor{Пользователь браузера с расписанием} может производить над ним операции, а именно он может:
			\begin{itemize}
				\item Экспортировать расписание;
					
				\item Изменить расписанием (добавить предмет, удалить предмет, изменить предмет, установить расписание пустым, импортировать расписание).
			\end{itemize}
			
			При этом после любых изменений расписание сохраняется в кэш браузера, заменяя предыдущее, и страница браузера перерисовывается с учётом изменений.
	\section{Нефункциональные требования}
		Клиентские части (расширение для браузера) должны быть совместимы между собой: любой пользователь должен иметь возможность импортировать расписание, экспортированное другим пользователем.
