\chapter{Известные проблемы}
	\label{chapter6}
	\phantomsection

	\section{Отсутствие поддержки JWT}
		\begin{tabularx}{\textwidth}{
				| >{\centering\arraybackslash\hsize=4cm}X
				| >{\centering\arraybackslash}X
				|}
			\hline
			\textbf{Проблема} &  Для доступа к данным API сервера не требуется авторизация \\\hline
			\textbf{Ранг} &  3 (низкий) \\\hline
			\textbf{Влияние на проект} &  Любой пользователь сети интернета может получать данные с сервера, что увеличивает нагрузку на него. Как следствие, расширение для браузера может начать работать медленнее \\\hline
			\textbf{Пути решения} &  Добавить поддержку JWT, чтобы доступ к серверу имелся лишь у авторизованных приложений (только у расширения) \\\hline
		\end{tabularx}
	\section{Отсутствие защиты от DOS и DDOS-атак}
		\begin{tabularx}{\textwidth}{
				| >{\centering\arraybackslash\hsize=4cm}X
				| >{\centering\arraybackslash}X
				|}
			\hline
			\textbf{Проблема} &  У сервера нет никакой защиты от DOS или DDOS-атак, поэтому при любой такой атаке он перестанет обрабатывать запросы расширения \\\hline
			\textbf{Ранг} &  5 (средний) \\\hline
			\textbf{Влияние на проект} &  При любой DOS или DDOS-атаке, расширения у клиентов перестанут полноценно работать \\\hline
			\textbf{Пути решения} &  Использовать интернет сервисы для защиты от DOS и DDOS-атак. Например, CloudFlare \\\hline
		\end{tabularx}
	\section{Невозможность работы расширения без сервера}
		\begin{tabularx}{\textwidth}{
				| >{\centering\arraybackslash\hsize=4cm}X
				| >{\centering\arraybackslash}X
				|}
			\hline
			\textbf{Проблема} &  Если не работает сервер, то расширение не может полноценно работать \\\hline
			\textbf{Ранг} &  7 (Высокий) \\\hline
			\textbf{Влияние на проект} &  При критических проблемах на стороне сервера расширение не могжет полноценно работать \\\hline
			\textbf{Пути решения} &  Избавиться от сервера и осуществлять парсинг сайта с расписанием НГУ на клиенте. Однако при добавлении нового клиентского фронтенда (например, мобильного приложения или телеграмм-бота) придётся дублировать логику, что может привести к расхождении спецификации и отсутствию совместимости клиентских фронтендов  \\\hline
		\end{tabularx}
	\section{Зависимость сервера и расширения от вёрстки сайта с расписанием НГУ}
		\begin{tabularx}{\textwidth}{
				| >{\centering\arraybackslash\hsize=4cm}X
				| >{\centering\arraybackslash}X
				|}
			\hline
			\textbf{Проблема} & Сервер и расширение напрямую зависят от вёрстки сайта с расписанием НГУ \\\hline
			\textbf{Ранг} &  8 (Высокий) \\\hline
			\textbf{Влияние на проект} & Если у сайта с расписанием НГУ поменяется вёрстка, то для полноценной работы системы придётся изменять сервер и отображение страниц в расширении \\\hline
			\textbf{Пути решения} &  Использовать API, которое используют сайты с расписанием НГУ, однако данное API если и существует, то не является открытым \\\hline
		\end{tabularx}
	\section{Невозможность работы расширения во всех браузерах без установки API на домен}
		\begin{tabularx}{\textwidth}{
				| >{\centering\arraybackslash\hsize=4cm}X
				| >{\centering\arraybackslash}X
				|}
			\hline
			\textbf{Проблема} & Политика многих современных браузеров такова, что из кода на JavaScript, запущенного на страничке, загруженной с помощью протокола HTTPS, не позволяется делать запросы к внешним сервисам по протоколу HTTP \\\hline
			\textbf{Ранг} &  10 (Высокий) \\\hline
			\textbf{Влияние на проект} & Если у API нет никакого домена (то есть доступ в сети интернет осуществляется только по IP-адресу), то во многих современных браузерах клиентская часть не может работать полноценно так как фактически не имеет доступа к серверной части \\\hline
			\textbf{Пути решения} & Установить API на домен в сети интернет \\\hline
		\end{tabularx}
