\newcommand{\Server}{Server}
\newcommand{\Extension}{Extension}

\newcommand{\Serverbf}{\textbf{\Server}}
\newcommand{\Extensionbf}{\textbf{\Extension}}

\chapter{Обзор архитектуры}
	\label{chapter4}
	\phantomsection

	Этот раздел описывает архитектуру проекта.

	\section{Подсистемы и компоненты проекта}
		Если не брать в расчёт компоненты сторонних производителей, то проект состоит из подсистем
		
		\begin{itemize}
			\item \Serverbf{};
			
			\item \Extensionbf{};
		\end{itemize}
			
		\subsection{\Server{}}
			\begin{figure}[H]
				\centering
				\def\svgwidth{\columnwidth}
				\input{server_classes.pdf_tex}
				\caption{Диаграмма классов \Serverbf}
			\end{figure}
		
			\begin{figure}[H]
				\centering
				\def\svgwidth{\columnwidth}
				\input{server_packages.pdf_tex}
				\caption{Диаграмма пакетов \Serverbf}
			\end{figure}
		
			\Serverbf{} представляет собой REST-сервер с API, который по запросу парсит сайт с расписанием НГУ и возвращает информацию о расписании группы, аудитории, преподавателе или временах начала пар. Компонента сделан в <<Чистой архитектуре>>
			
			\begin{itemize}
				\item \textit{model} --- представляет собой слой с основными dto предметной области;
				
				\item \textit{services} --- представляет собой слой бизнес-логику (манипулирование классами предметной области);
				
				\item \textit{controller} --- представляет собой слой адаптеров между бизнес-логикой и запросами/ответами сервера;
				
				\item \textit{routing} --- представляет собой слой, отвечающие за межсетевое взаимодействие (принятие запросов и отправка ответов);
				
				\item \textit{config-parsing} --- представляет собой пакет, отвечающий за парсинг конфигурации сервера.
			\end{itemize}
		\subsection{\Extension{}}
			\begin{figure}[H]
				\centering
				\def\svgwidth{\columnwidth}
				\input{ext_classes.pdf_tex}
				\caption{Диаграмма классов \Extensionbf}
			\end{figure}
			
			\begin{figure}[H]
				\centering
				\def\svgwidth{\columnwidth}
				\input{ext_packages.pdf_tex}
				\caption{Диаграмма пакетов \Extensionbf}
			\end{figure}
			
			\Extensionbf{} представляет собой расширение для браузеров, являющееся фронтендом проекта. Архитектурно \Extensionbf{} реализован с использованием паттерна MVC.
			
			\begin{itemize}
				\item \textit{model} --- представляет собой слой dto и бизнес-логики;
				
				\item \textit{view} --- представляет собой слой представления, то есть то что пользователь непосредственно видит и с чем взаимодействует (сообщения об ошибках, формы для ввода, ...);
				
				\item \textit{controller} --- представляет собой слой классов, отвечающих за обработку пользовательских действий (нажатия на кнопки, изменения предметов, ...);
			\end{itemize}
	\section{Компоненты сторонних производителей}	
		\begin{itemize}
			\item \textbf{python-dotenv} --- библиотека под Python для парсинга файлов .env. Используется в подсистеме \Serverbf{} для парсинга начальной конфигурации.
			
			\item \textbf{beautifulsoup4} ---- библиотека под Python для парсинга текстовых документов. Используется в подсистеме \Serverbf{} для парсинга информации из HTML-документов.
			
			\item \textbf{fastapi} --- библиотека под Python для создания своего API. Используется в подсистеме \Serverbf{} для создания конечных точек.
			
			\item \textbf{Toastify} --- библиотека под JS, предоставляющая способ показа различных уведомлений в браузерах. Используется в подсистеме \Extensionbf{}.
		\end{itemize}
	\section{Схема развёртывания проекта}
		\subsubsection{Развёртывание \Extensionbf{}}
			Сборка расширения в файлы bundle.js и manifest.json и дальнейшее добавление в каталог расширений современных браузеров.
		\subsubsection{Развёртывание \Server{}}
			При изменениях в коде GitHub Actions запускает предварительно настроенный workflow. Виртуальная 	машина на GitHub Actions создает Docker-образы из актуального кода, собранные образы помещаются в Docker Hub. После этого останавливаются и удаляются старые Docker-контейнеры на VPS, а контейнеры 
			с новыми Docker-образами создаются и запускаются, после чего \Serverbf{} автоматически перезапускаются.
